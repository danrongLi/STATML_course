\documentclass[11pt]{article}

\usepackage{amsmath}
\usepackage{amsthm}
\usepackage{amssymb}
\usepackage[shortlabels]{enumitem}
\usepackage{fullpage}
\usepackage{graphicx}
\usepackage{natbib}
\usepackage[usenames,svgnames]{xcolor}
\usepackage[pdftex,
            colorlinks=true,
            linkcolor=NavyBlue,
            citecolor=NavyBlue,
            urlcolor=NavyBlue,
            breaklinks=true]{hyperref}

\title{Solution for Homework \#0}

\author{Danrong Li\\
\href{dl4111@nyu.edu}{dl4111@nyu.edu}
}

\begin{document}

\maketitle



\section*{Solution to Question 1}

\subsection*{Solution to Part (a)}

\indent

P(first bin is empty) = (1-P(first ball in first bin))$^m$

= $(1-\frac{1}{n})^m$

= $(\frac{n-1}{n})^m$

\subsection*{Solution to Part (b)} 

\indent

E(number of empty bins) = P(bin is empty) * n

= $(\frac{n-1}{n})^m*n$

\subsection*{Solution to Part (c)}

\indent

$\lim_{n\to\infty}\frac{a_{n,n}}{n}=\lim_{n\to\infty}(\frac{n-1}{n})^n$ = f(n)

$\ln$(f(n)) = $\lim_{n\to\infty}n*\ln(\frac{n-1}{n})$

\hspace{1.3cm} = $\lim_{n\to\infty}\frac{\ln(\frac{n-1}{n})}{\frac{1}{n}}$

\hspace{1.3cm} = $\lim_{n\to\infty}\frac{\frac{d}{dn}*(\ln(1-\frac{1}{n}))}{\frac{d}{dn}*\frac{1}{n}}$

\hspace{1.3cm} = $\lim_{n\to\infty}\frac{\frac{1}{1-\frac{1}{n}}*(-1)*(-1)*n^{-2}}{-1*n^{-2}}$

\hspace{1.3cm} = $\lim_{n\to\infty}\frac{1}{1-\frac{1}{n}}*(-1)$

\hspace{1.3cm} = -1

Thus we have $\ln$f(n) = -1, and it means that f(n) = $e^{-1}$

Then it proves that $\lim_{n \to \infty} \frac{a_{n,n}}{n} = \frac{1}{e}$


\section*{Solution to Question 2}

\subsection*{Solution to Part (a)}

    \indent
    
    $AB$ is $O(n)$
    
    The shape of $AB$ is $d*d$, for each element in $AB$, it is calculated by a summation of n numbers, and each number is a product. So $AB$ would need $n*d*d$ calculations in total, and it can be concluded as $O(n)$.
    
    $BC$ is $O(n^2)$
    
    The shape of $BC$ is $n*n$, for each element in $BC$, it is calculated by a summation of d numbers, and each number is a product. So $BC$ would need $d*n*n$ calculations in total, and it can be concluded as $O(n^2)$.
    
    $CD$ is $O(n)$
    
    The shape of $CD$ is $d*1$, for each element in $CD$, it is calculated by a summation of n number, and each number is a product. So $CD$ would need $n*d$ calculations in total, and it can be concluded as $O(n)$.
    
\subsection*{Solution to Part (b)}

\indent
    
    $ABCD$ is $O(n)$
    
    First calculate $AB$, which takes $O(n)$, and we have a matrix of shape $d*d$. Then we calculate $CD$, which takes $O(n)$, and we have a matrix of shape $d*1$. Then we multiply the resultant 2 matrices together to get the product of $ABCD$, and this process would do calculations $d*d$ times, which can be concluded as $O(1)$ time complexity and the final product shape is $d*1$. Thus, this algorithm would have 3 steps which have the time complexity of $O(n)$, $O(n)$, $O(1)$ respectively, which makes the algorithm time complexity of $O(n)$.
    


\end{document}
