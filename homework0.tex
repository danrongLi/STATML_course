\documentclass[11pt]{article}

\usepackage{amsmath}
\usepackage{amsthm}
\usepackage{amssymb}
\usepackage[shortlabels]{enumitem}
\usepackage{fullpage}
\usepackage{graphicx}
\usepackage{natbib}
\usepackage[usenames,svgnames]{xcolor}
\usepackage{subcaption}
\usepackage{tikz}
\usepackage[pdftex,
            colorlinks=true,
            linkcolor=NavyBlue,
            citecolor=NavyBlue,
            urlcolor=NavyBlue,
            breaklinks=true]{hyperref}


% TikZ libraries
\usetikzlibrary{arrows.meta}
\usetikzlibrary{matrix}

% Theorem-like environments
\newtheorem{theorem}{Theorem}
\newtheorem{lemma}[theorem]{Lemma}
\newtheorem{definition}[theorem]{Definition}
\newtheorem{proposition}[theorem]{Proposition}
\newtheorem{corollary}[theorem]{Corollary}

% Mathematical symbols and operators
\newcommand{\N}{\mathbb{N}}  % Set of natural numbers
\newcommand{\Z}{\mathbb{Z}}  % Set of integers
\newcommand{\R}{\mathbb{R}}  % Set of real numbers

\DeclareMathOperator{\Exp}{\mathbf{E}}

\title{Statistical and Computational Foundations of Machine Learning \\
NYU Tandon School of Engineering \\
Spring 2022, Homework \#0
}

\author{D\'avid P\'al}

\date{January 25, 2022}

\begin{document}

\maketitle

Submit your solutions by email to
\href{mailto:david.pal@nyu.edu}{david.pal@nyu.edu} with subject "Homework \#0".
Use LaTeX and submit your solution as a PDF file. Use the template provided on
the course web page. Email your solutions by \textbf{23:59:59 EST, February 1,
2022}.

Collaboration is allowed on homework, but solutions must be written
independently. If you use external sources (books, online documents), list them
in your solutions. See homework policy on the course webpage for more details.

\section*{Question 1}

Suppose that we throw $m$ balls into $n$ bins independently at random.

\begin{enumerate}[(a)]

\item \emph{(1 point)} Calculate the probability that the first bin is empty.

\item \emph{(1 point)} Calculate the expected number of empty bins.

\item \emph{(1 point)} Let $a_{m,n}$ be the expected number of empty bins. Prove
that
$$
\lim_{n \to \infty} \frac{a_{n,n}}{n} = \frac{1}{e} \: .
$$
\end{enumerate}

\textbf{Answer} 
\begin{enumerate}[(a)]

\item P(first bin is empty) = (1-P(first ball in first bin))$^m$

= $(1-\frac{1}{n})^m$

= $(\frac{n-1}{n})^m$

\item E(number of empty bins) = P(bin is empty) * n

= $(\frac{n-1}{n})^m*n$

\item $\lim_{n\to\infty}\frac{a_{n,n}}{n}=\lim_{n\to\infty}(\frac{n-1}{n})^n$ = f(n)

$\ln$(f(n)) = $\lim_{n\to\infty}n*\ln(\frac{n-1}{n})$

\hspace{1.3cm} = $\lim_{n\to\infty}\frac{\ln(\frac{n-1}{n})}{\frac{1}{n}}$

\hspace{1.3cm} = $\lim_{n\to\infty}\frac{\frac{d}{dn}*(\ln(1-\frac{1}{n}))}{\frac{d}{dn}*\frac{1}{n}}$

\hspace{1.3cm} = $\lim_{n\to\infty}\frac{\frac{1}{1-\frac{1}{n}}*(-1)*(-1)*n^{-2}}{-1*n^{-2}}$

\hspace{1.3cm} = $\lim_{n\to\infty}\frac{1}{1-\frac{1}{n}}*(-1)$

\hspace{1.3cm} = -1

Thus we have $\ln$f(n) = -1, and it means that f(n) = $e^{-1}$

Then it proves that $\lim_{n \to \infty} \frac{a_{n,n}}{n} = \frac{1}{e}$

\end{enumerate}

\newpage
\section*{Question 2}

Let $n,d$ be positive integers such that $n \gg d$. Let $A,B,C,D$ be matrices of
dimensions $d \times n$, $n \times d$, $d \times n$, $n \times 1$ respectively.

\begin{enumerate}[(a)]

\item \emph{(1 point)} Suppose we run the standard matrix multiplication
algorithm to compute the products $AB$, $BC$, $CD$. Express the asymptotic time
complexity of each of the three computations using big-O notation! Provide
explanation for your answer.

\item \emph{(1 point)} How do you compute the product $ABCD$? Describe your
algorithm in detail. Try to come with the algorithm with as low time complexity
as you can. Express the time complexity of your algorithm using big-O notation!

\end{enumerate}

\textbf{Answer}

\begin{enumerate}[(a)]
    \item $AB$ is $O(n)$
    
    The shape of $AB$ is $d*d$, for each element in $AB$, it is calculated by a summation of n numbers, and each number is a product. So $AB$ would need $n*d*d$ calculations in total, and it can be concluded as $O(n)$.
    
    $BC$ is $O(n^2)$
    
    The shape of $BC$ is $n*n$, for each element in $BC$, it is calculated by a summation of d numbers, and each number is a product. So $BC$ would need $d*n*n$ calculations in total, and it can be concluded as $O(n^2)$.
    
    $CD$ is $O(n)$
    
    The shape of $CD$ is $d*1$, for each element in $CD$, it is calculated by a summation of n number, and each number is a product. So $CD$ would need $n*d$ calculations in total, and it can be concluded as $O(n)$.
    
    \item $ABCD$ is $O(n)$
    
    First calculate $AB$, which takes $O(n)$, and we have a matrix of shape $d*d$. Then we calculate $CD$, which takes $O(n)$, and we have a matrix of shape $d*1$. Then we multiply the resultant 2 matrices together to get the product of $ABCD$. This process would take $O(1)$ and the final product shape is $d*d$. 
    
    
    
\end{enumerate}

\end{document}
